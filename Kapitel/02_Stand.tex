% Vorlage für WABs der Provadis Hochschule
% basiert auf "Universität Ulm Praktikumsbericht Vorlage"
% von Max Sch.
% CC BY 4.0
%
% 02_Stand.tex

\chapter{Stand der Forschung}

In diesem Kapitel wird besonders intensiv mit der Literatur gearbeitet. Entsprechend werden hier sehr viele Zitate vorkommen. Ich werde die Quellenhinweise sofort einbauen, damit ich sie nicht vergesse. 

Dabei beachte ich die Vorgaben der Provadis School of International Management and Technology, die in den „Anforderungen an wissenschaftliche Hausarbeiten“ beschrieben sind.

Die Literatur wird in einem Verzeichnis am Ende der Arbeit aufgeführt.
Ebenso werden Quellen (Interviews, Internetseiten) in einem gesonderten Quellenverzeichnis genannt.

Literaturquellen werden in der Datei Referenzen.bib im BibTeX-Format eingetragen und im Text~\cite{lustigeCitation} zitiert. Gut gepflegte BibTeX-Einträge finden sich etwa in der DBLP\footnote{\url{https://dblp.uni-trier.de}} und in den Online-Bibliotheken großer Wissenschaftsverlage, wie der ACM Digital Library\footnote{\url{https://dl.acm.org}}, IEEE Xplore\footnote{\url{https://ieeexplore.ieee.org}}, Springer Link\footnote{\url{https://link.springer.com}}, Elsevier ScienceDirect\footnote{\url{https://www.elsevier.com/de-de/products/sciencedirect}}, uvw.

Weitere Verzeichnisse sind alphabetisch zu sortieren und besonders sorgfältig anzufertigen.